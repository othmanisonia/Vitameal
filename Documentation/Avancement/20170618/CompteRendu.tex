%-*- coding: utf-8 -*-
\documentclass[11pt,a4paper,french,twoside,openright]{article}
\usepackage[utf8]{inputenc}
\usepackage[T1]{fontenc}
\usepackage{graphicx}%pour insérer images et pdf entre autres
	\graphicspath{{images/}}%pour spécifier le chemin d'accès aux images
\usepackage[left=2.5cm,right=2.5cm,top=2.5cm,bottom=2.5cm]{geometry}%réglages des marges du document selon vos préférences ou celles de votre établissemant
\usepackage[Lenny]{fncychap}%pour de jolis titres de chapitres voir la doc pour d'autres styles.
\usepackage{babel}
\usepackage[babel=true]{csquotes} % csquotes va utiliser la langue définie dans babel

\usepackage{fancyhdr}%pour les entêtes et pieds de pages
	\setlength{\headheight}{14.2pt}% hauteur de l'entête
        \chead{\textbf{VITAMEAL}}
        \lhead{}
        \rhead{}
	\cfoot{Formation Ingénieur Informatique en alternance - Première année}
	\lfoot{\textbf{CNAM}}%
  	\rfoot{\textbf{\thepage/\pageref{LastPage}}}
 	\renewcommand{\headrulewidth}{0.4pt}%trait horizontal pour l'entête
  	\renewcommand{\footrulewidth}{0.4pt}%trait horizontal pour les pieds de pages

\usepackage[french]{nomencl}
\makenomenclature
\usepackage{hyperref}
\usepackage{mathtools}
\begin{document}
\pagestyle{fancy}

\begin{center}\bfseries\Huge
COMPTE RENDU DE TÉLÉCONFÉRENCE
\end{center}

\textbf{Du      :} dimanche 18/06/2017 à 20h30

\textbf{Objet   :} Avancement projet VITAMEAL

\textbf{Présents:} Nicolas SYMPHORIEN, Sonia OTHMANI, Jean-Félix BENITEZ

\textbf{Absent :} Personne.

\textbf{Diffusion:} Nicolas SYMPHORIEN, Sonia OTHMANI, Jean-Félix BENITEZ

\hrulefill

\section{Cas d'utilisations}
Les diagrammes de cas d'utilisations, avec leurs descriptions et avec les diagrammes de séquences sont terminés.
Sonia et Nicolas, doivent renseigner la partie évolution du rapport, sur le travail qu'ils ont remontés.
La prochaine étape est la conception de ces cas d'utilisation avant le codage.

Nous allons \enquote{tagger} le dépôt avant de commencer les développements $\rightarrow$ Jean-Félix

\section{Modèle du domaine}
Le modèle du domaine tel qu'il a été fait est adopté; il faut juste remplacer \enquote{Modéle Conceptuel de Données} par \enquote{Modèle Logique de Données} $\rightarrow$ Jean-Félix

\section{Environnement de développement}
Sonia doit finaliser l'installation de son environnement de développement.

\section{Architecture logicielle}
Elle doit être affinée; Nicolas regarde cela en tâche de fond.
La mise en oeuvre d'\emph{Hibernate} est plus délicate que prévue. Doit-on continuer à l'utiliser ou bien doit-on passer à quelque chose de plus simple, comme le modèle \emph{DAO} ?
Nous nous efforcerons de répondre à cette question dans le courant de la semaine qui viens.

\section{Prochain rendez-vous}
\textbf{mercredi 21/06/2017 20h30 après les cours.}

\label{LastPage}
\end{document}
