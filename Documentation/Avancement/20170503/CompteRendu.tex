%-*- coding: utf-8 -*-
\documentclass[11pt,a4paper,french,twoside,openright]{article}
\usepackage[utf8]{inputenc}
\usepackage[T1]{fontenc}
\usepackage{graphicx}%pour insérer images et pdf entre autres
	\graphicspath{{images/}}%pour spécifier le chemin d'accès aux images
\usepackage[left=2.5cm,right=2.5cm,top=2.5cm,bottom=2.5cm]{geometry}%réglages des marges du document selon vos préférences ou celles de votre établissemant
\usepackage[Lenny]{fncychap}%pour de jolis titres de chapitres voir la doc pour d'autres styles.
\usepackage{babel}
\usepackage[babel=true]{csquotes} % csquotes va utiliser la langue définie dans babel

\usepackage{fancyhdr}%pour les entêtes et pieds de pages
	\setlength{\headheight}{14.2pt}% hauteur de l'entête
        \chead{\textbf{VITAMEAL}}
        \lhead{}
        \rhead{}
	\cfoot{Formation Ingénieur Informatique en alternance - Première année}
	\lfoot{\textbf{CNAM}}%
  	\rfoot{\textbf{\thepage/\pageref{LastPage}}}
 	\renewcommand{\headrulewidth}{0.4pt}%trait horizontal pour l'entête
  	\renewcommand{\footrulewidth}{0.4pt}%trait horizontal pour les pieds de pages

\usepackage[french]{nomencl}
\makenomenclature
\usepackage{hyperref}
\usepackage{mathtools}
\begin{document}
\pagestyle{fancy}

\begin{center}\bfseries\Huge
COMPTE RENDU DE TÉLÉCONFÉRENCE
\end{center}

\textbf{Du      :} mercredi 03/05/2017 à 20h30

\textbf{Objet   :} Avancement projet VITAMEAL

\textbf{Présents:} Nicolas SYMPHORIEN, Sonia OTHMANI, Jean-Félix BENITEZ

\textbf{Absent :} Personne.

\textbf{Diffusion:} Nicolas SYMPHORIEN, Sonia OTHMANI, Jean-Félix BENITEZ

\hrulefill

\section{Profil patient}
Nous avons convenu que le profil patient est constituté des éléments suivants:
\begin{itemize}
\item age
\item poids
\item taille
\item allergie
\item contre-indications
\item diabète $\rightarrow$ pas de sucre !
\item colestérol $\rightarrow$ pas de matières grasses !
\item hypertension artérielle $\rightarrow$ pas de sel !
\end{itemize}

\section{Cas d'utilisations}
Dans un premier temps nous allons définir les cas d'utilisations suivants:
\begin{enumerate}
\item Renseigner un profil patient $\rightarrow$ Sonia
  \begin{itemize}
   \item REQ\_09020
   \item REQ\_10040
  \end{itemize}
\item Composer un nouveau plat $\rightarrow$ Nicolas
  \begin{itemize}
   \item REQ\_01060
   \item REQ\_01070
   \item REQ\_01080
  \end{itemize}
\item Générer les menus $\rightarrow$ Jean-Félix
  \begin{itemize}
   \item REQ\_01000
   \item REQ\_10020
   \item REQ\_10050
   \item petit déjeuner
   \begin{itemize}
     \item REQ\_01010
     \item REQ\_01020
     \item REQ\_01030
     \item REQ\_01040
   \end{itemize}
   \item déjeuner
   \begin{itemize}
     \item REQ\_01050
   \end{itemize}
   \item diner
   \begin{itemize}
     \item REQ\_01050
   \end{itemize}
  \end{itemize}
\item Publier les menus $\rightarrow$ Nicolas
  \begin{itemize}
   \item REQ\_05010
   \item mettre à disposition du service restauration
   \item Assembler les menus
   \begin{itemize}
     \item REQ\_07000
     \item REQ\_10030
   \end{itemize}
  \end{itemize}
\end{enumerate}
Chacun remontera son travail directement dans le rapport final sous GitHub, pour samedi soir prochain.
Nous avons convenu d'utiliser \href{https://eclipse.org/papyrus/}{Papyrus} pour les diagrammes UML.

\section{Sprint}
Nous mettrons en place les bases de notre travail sur les sprints lors de la prochaine téléconférence: \textbf{samedi 6 mai à 20h30} afin d'avoir le premier srpint parfaitement défini et présentable pour lundi 8 mai au soir.

\label{LastPage}
\end{document}
