%-*- coding: utf-8 -*-
\documentclass[11pt,a4paper,french,twoside,openright]{article}
\usepackage[utf8]{inputenc}
\usepackage[T1]{fontenc}
\usepackage{graphicx}%pour insérer images et pdf entre autres
	\graphicspath{{images/}}%pour spécifier le chemin d'accès aux images
\usepackage[left=2.5cm,right=2.5cm,top=2.5cm,bottom=2.5cm]{geometry}%réglages des marges du document selon vos préférences ou celles de votre établissemant
\usepackage{babel}
\usepackage[babel=true]{csquotes} % csquotes va utiliser la langue définie dans babel
\usepackage[french]{nomencl}
\usepackage{hyperref}

\begin{document}
\section{EXIGENCES}\label{exigences}
Elles sont stockées dans un fichier XML dont la structure est décrite ci-dessous.
\subsection{Composition}\label{composition}
Chaque exigence est composée de 10 champs:
\begin{description}
\item[numero:] Numéro formé comme suit: REQ\_xxxxx, où x est un chiffre de 0 à 9
\item[titre:] Titre ou description courte
\item[corps:] Expression de l'exigence
\item[type:] Type de l'exigence: Utilisateur - Métier - Technique - Fonctionnelle - Non fonctionnelle - Contrainte - Ergonomie - Robustesse - Performance - Sécurité
\item[origine:] D'où vient une exigence ? (Quel besoin cette exigence couvre-t-elle ? Pourquoi a-t-on conçu la solution de cette manière et quelles étaient les autres possibilités ?)
\item[version:] ou niveau de maturité, voir \href{http://users.polytech.unice.fr/~hugues/GL/CMM/cmm.html}{Capability Maturity Model} - Initial - Reproductible - Défini - Maîtrisé - Optimisé
\item[priorite:] Priorité selon la méthode MoSCoW - Must - Should - Could - Won't
\item[validee:] L'exigences a-t-elle été validée ? (Oui / Non)
\item[lien:] il peut y en avoir plusieurs; ils sont regroupés dans l'élément ``Liens''.
\item[test:] Définition du test qui validera l'exigence.
\end{description}

\subsection{Edition}\label{edition}

Elle peut ce faire avec \enquote{Eclipse}, qui permet la validation (par
rapport au schéma) au fur et à mesure de la saisie. Une commodité dans
\enquote{Eclipse} est la liste des énumération disponible sur un élément, en
tapant Ctrl + Espace dans l'élément.

\subsection{Extraction}\label{extraction}

L'export des informations contenues dans le fichier XML ce fait à l'aide de scripts XSLT:
\begin{description}
\item[Exigences.xsl:] génère le fichier \enquote{Exigences.tex} qui sera inclu automatiquement dans le rapport du projet.
\item[Exigences\_csv.xsl:] génère le fichier \enquote{Exigences.csv} qui peut être lu par n'importe quel tableur.
\item[Exigences\_html.xsl:] génère le fichier \enquote{Exigences.html} qui peut être lu par n'importe quel navigateur internet.
\end{description}

J'utilise le processeur XSLT de \href{http://www.saxonica.com/download/opensource.xml}{Saxonica}. Ce processeur fonctionne avec Java et ne s'installe pas, il suffit juste de le désarchiver dans un dossier.

Le fichier \enquote{gen.cmd} permet de lancer la génération; avant il faut mettre à jour la variable d'environnement
\enquote{XsltProcPath} avec le chemin d'accès au processeur \enquote{Saxon}.

\end{document}
