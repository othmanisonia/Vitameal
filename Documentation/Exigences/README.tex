\section{EXIGENCES}\label{exigences}

\subsection{Composition}\label{composition}
Chaque exigence est composée de 10 champs:
\begine{itemize}
\item \textbf{numéro:} formé comme suit REQ\_xxxx, où x est un chiffre de 0 à 9
\item \textbf{Titre:} titre ou description courte
\item \textbf{Corps:} expression de l'exigence
\item \textbf{Type:} type de l'exigence: Utilisateur - Métier - Technique - Fonctionnelle - Non fonctionnelle - Contrainte - Ergonomie - Robustesse - Performance - Sécurité
\item \textbf{Origine:} D'où vient une exigence ? (Quel besoin cette exigence couvre-t-elle ? Pourquoi a-t-on conçu la solution de cette manière et quelles étaient les autres possibilités ?)
\item \textbf{Version:} ou niveau de maturité, voir \href{http://users.polytech.unice.fr/~hugues/GL/CMM/cmm.html}{Capability Maturity Model} - Initial - Reproductible - Défini - Maîtrisé - Optimisé
\item \textbf{Priorité:} priorité selon la méthode MoSCoW - Must - Should - Could - Won't
\item \textbf{Validée:} l'exigences a-t-elle été validée ? (Oui / Non)
\item \textbf{Lien:} il peut y en avoir plusieurs; ils sont regroupés dans l'élément ``Liens''.
\item \textbf{Test:} Définition du test qui validera l'exigence.
\end{itemize}

\subsection{Edition}\label{edition}

Elle peut ce faire avec \enquote{Eclipse}, qui permet la validation (par
rapport au schéma) au fur et à mesure de la saisie. Une commodité dans
\enquote{Eclipse} est la liste des énumération disponible sur un élément, en
tapant Ctrl + Espace dans l'élément.

\subsection{Extraction}\label{extraction}

Elle ce fait à l'aide du script XSLT \enquote{Exigences.xsl}, lequel génère le
fichier \enquote{Exigences.tex} qui sera inclu automatiquement dans le rapport
du projet. J'utilise le processeur XSLT de
\href{http://www.saxonica.com/download/opensource.xml}{Saxonica} qui
fonctionne avec Java. Le fichier \enquote{gen.cmd} permet de lancer la
génération; avant il faut mettre à jour la variable d'environnement
\enquote{XsltProcPath} avec le chemin d'accès au processeur \enquote{Saxon}. Le
processeur Saxon ne s'installe pas, il suffit juste de le désarchiver
dans un dossier.
