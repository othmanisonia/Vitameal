%-*- coding: utf-8 -*-
\textcolor[RGB]{46, 116, 181}{\chapter{Initialisation}}
Quelle que soit l’importance des avancées scientifiques et technologiques, c’est le travail des
professionnels de santé qui détermine la qualité et l’efficacité des soins. Dans ce contexte, les soins
nutritionnels, qui portent sur l’évaluation de l’état nutritionnel et l’accompagnement alimentaire des
patients hospitalisés, en interaction étroite avec l’équipe de soin, ne font pas exception. Pour ce
faire, les diététiciens développent des actions de complexité variable, tant au niveau des services de
soins que du système de restauration.

Simultanément, les professionnels doivent faire face à de nouveaux défis, dus aux modifications des
profils épidémiologiques, démographiques et sociaux des populations, ce qui exige la mise en place
de nouvelles compétences et la reconfiguration des stratégies d’action. Pour les diététiciens du
secteur hospitalier, elles ont pour conséquences de nouvelles exigences mentales et surtout
cognitives.

Le niveau de développement industriel de la filière alimentaire française allège la charge de travail
technique des diététiciens, non seulement en ce qui concerne la diversité de matières premières,
mais également dans le domaine du contrôle \enquote{qualité}, tout au long de la chaîne de production. De
la même façon, les nouveaux concepts de production en restauration collective, caractérisés par
l’utilisation de produits pré élaborés et l’innovation technologique des équipements, gagnent
visiblement du terrain dans le secteur hospitalier français.

\section{Définition du problème}
L'élaboration de menus dans un hôpital pour la restauration des patients
est une tâche complexe, et doit tenir compte des différentes pathologies
rencontrées. Faute de moyens (temps et argent) seules quelques grandes
lignes de restauration sont retenues; alors qu'idéalement, chaque
patient devrait pourvoir avoir un repas adapté à sa pathologie.

\section{Vision du projet}
\subsection{Solution envisagée}
Le projet Vitameal a pour objectif de faire correspondre au mieux la planification des régimes et des
prescriptions diététiques aux repas réellement servis au patient. Il consiste en un outil interfaçant la
gestion de production, la prise de commande et le suivi nutritionnel des repas.

\subsection{Périmètre}
C'est un diététicien qui renseigne le profil diététique des patients,
sous les directives des médecins. C'est aussi un diététicien qui élabore
les menus des patients. L'outil élaborera donc
les menus par filtrage des produits correspondants aux profils
diététiques des patients. Pour des raisons de simplifications, nous nous limiterons dans ce projet aux seuls patients adolescents et adultes, à l'exclusion des personnes agées.
\begin{figure}[H]
\label{Modelisation_du _probleme}
  \centering
      \includegraphics[width=0.75\textwidth]{problem_model} %
\caption{Modélisation du problème}
\end{figure}

\section{Analyse des exigences}
\subsection{Partie prenantes}
\begin{itemize}
\item Participantes~: les diététiciens, le service restauration
\item Concernés~: les médecins, la direction (budget)
\item Impactées~: les patients
\end{itemize}

\subsection{Les besoins}
\begin{description}
\item[] En tant que diététicien, j’ai besoin de :
\begin{description}
 \item[N001:] pouvoir renseigner le profil diététique des patients, afin qu’ils
 puissent bénéficier de menus adaptés.
 \item[N002:] pouvoir élaborer les menus des 3 repas journaliers
 (petit-déjeuner, déjeuner et dîner dont la composition est décrite en annexe
 \ref{annexeA}), de façon automatique, en tenant compte de grammages dépendant du type d'aliments et de la tranche d'age (Document \ref{docNutrition}, Annexe 2).
 \item[N003:] pouvoir saisir des plats et leur composition.
 \item[N004:] élaborer des menus selon les fréquences de service, selon le
 document \ref{docNutrition}, Annexe 4.
 \item[N005:] classer chaque aliment dans une des catégories d’aliments citée
 dans les tables de grammages du document \ref{docNutrition}, Annexes 2 et 4.
\end{description}
\item[N006:] En tant qu’administrateur du site internet de l’hôpital, j’ai
besoin de récupérer le menu de la semaine, afin pouvoir l’intégrer au site.
\item[N007:] En tant que médecin, j’ai besoin de consulter les profils
diététiques des patients admis, pour les  valider.
\item[N008:] En tant que cuisinier du service restauration, j’ai besoin de
consulter les menus élaborés, afin de pouvoir les préparer et prévoir les ingrédients à commander.
\item[N009:] En tant qu’agent de restauration hospitalière,  j’ai besoin de
connaître les menus de chaque patient, afin de pouvoir assembler les plateaux repas.
\end{description}
\subsection{Les contraintes}
\begin{description}
\item[N010:] Les médecins doivent pouvoir vérifier / valider les profils
diététiques des patients.
\item[N011:] La direction fixe un budget maximum par menu.
\end{description}

\subsection{Exigences}

\rowcolors{1}{}{}

\begin{table}[!h]

\begin{tabular}{|p{60mm}p{100mm}|}

\hline

\multicolumn{2}{|l|}{\textbf{REQ\_0100:} 3 repas} \\ \hline

\emph{Type:} Métier & \emph{Liens:} REQ\_0101  \\

\emph{Origine:}  & \emph{Validé:} Non \\

\emph{Version:} Initial & \emph{Test:}  \\

\emph{Priorité:} Must & \\ \hline

\multicolumn{2}{|p{16cm}|}{Le système doit permettre de concevoir les 3 repas (petit-déjeuner, déjeuner, souper) d'une journée.} \\ \hline

\end{tabular}

\end{table}



\begin{table}[!h]

\begin{tabular}{|p{60mm}p{100mm}|}

\hline

\multicolumn{2}{|l|}{\textbf{REQ\_0101:} Petit déjeuner} \\ \hline

\emph{Type:} Métier & \emph{Liens:}  \\

\emph{Origine:}  & \emph{Validé:} Non \\

\emph{Version:} Initial & \emph{Test:}  \\

\emph{Priorité:} Should & \\ \hline

\multicolumn{2}{|p{16cm}|}{Le système doit permettre de concevoir un petit-déjeuner composé d'une boisson, d'un aliment céréalier, d'un produit laitier et d'un fruit.} \\ \hline

\end{tabular}

\end{table}



\begin{table}[!h]

\begin{tabular}{|p{60mm}p{100mm}|}

\hline

\multicolumn{2}{|l|}{\textbf{REQ\_0102:} Éléments petit déjeuner} \\ \hline

\emph{Type:} Métier & \emph{Liens:}  \\

\emph{Origine:}  & \emph{Validé:} Non \\

\emph{Version:} Initial & \emph{Test:}  \\

\emph{Priorité:} Must & \\ \hline

\multicolumn{2}{|p{16cm}|}{Le système doit permettre de rajouter au petit déjeuner un élément lipidique, sucré ou protodique.} \\ \hline

\end{tabular}

\end{table}



\begin{table}[!h]

\begin{tabular}{|p{60mm}p{100mm}|}

\hline

\multicolumn{2}{|l|}{\textbf{REQ\_0103:} Éléments non diététiques} \\ \hline

\emph{Type:} Métier & \emph{Liens:}  \\

\emph{Origine:}  & \emph{Validé:} Non \\

\emph{Version:} Initial & \emph{Test:}  \\

\emph{Priorité:} Should & \\ \hline

\multicolumn{2}{|p{16cm}|}{Le système doit avertir l'utilisateur de l'usage d'élément non diététique dans un petit déjeuner.} \\ \hline

\end{tabular}

\end{table}



\begin{table}[!h]

\begin{tabular}{|p{60mm}p{100mm}|}

\hline

\multicolumn{2}{|l|}{\textbf{REQ\_0104:} Fréquence éléments non diététiques} \\ \hline

\emph{Type:} Métier & \emph{Liens:}  \\

\emph{Origine:}  & \emph{Validé:} Non \\

\emph{Version:} Initial & \emph{Test:}  \\

\emph{Priorité:} Should & \\ \hline

\multicolumn{2}{|p{16cm}|}{Le système doit vérifier que la fréquence de l'usage d'élément non diététique des petits déjeuners ne dépasse pas 3 repas sur 20, il avertit l'utilisateur si c'est le cas.} \\ \hline

\end{tabular}

\end{table}



\begin{table}[!h]

\begin{tabular}{|p{60mm}p{100mm}|}

\hline

\multicolumn{2}{|l|}{\textbf{REQ\_0105:} Composition déjeuner} \\ \hline

\emph{Type:} Métier & \emph{Liens:}  \\

\emph{Origine:}  & \emph{Validé:} Non \\

\emph{Version:} Initial & \emph{Test:}  \\

\emph{Priorité:} Must & \\ \hline

\multicolumn{2}{|p{16cm}|}{Le système de concevoir un déjeuner et souper composés de 4 ou cinq composantes parmi : entrée, plat protodique, garniture, produit, laitier desserts (selon le tableau TODO ref) + de l'eau et du pain.} \\ \hline

\end{tabular}

\end{table}



\begin{table}[!h]

\begin{tabular}{|p{60mm}p{100mm}|}

\hline

\multicolumn{2}{|l|}{\textbf{REQ\_0106:} Ajout de plats} \\ \hline

\emph{Type:} Métier & \emph{Liens:}  \\

\emph{Origine:}  & \emph{Validé:} Non \\

\emph{Version:} Initial & \emph{Test:}  \\

\emph{Priorité:} Must & \\ \hline

\multicolumn{2}{|p{16cm}|}{Le système doit permettre d'ajouter des plats et leur définition dans la listes des plats pouvant être préparés.} \\ \hline

\end{tabular}

\end{table}



\begin{table}[!h]

\begin{tabular}{|p{60mm}p{100mm}|}

\hline

\multicolumn{2}{|l|}{\textbf{REQ\_0107:} Description d'un plat} \\ \hline

\emph{Type:} Métier & \emph{Liens:}  \\

\emph{Origine:}  & \emph{Validé:} Non \\

\emph{Version:} Initial & \emph{Test:}  \\

\emph{Priorité:} Must & \\ \hline

\multicolumn{2}{|p{16cm}|}{Le système doit permettre la description d'un plat avec sa liste d'ingrédients et les quantités nécessaires à sa réalisation.} \\ \hline

\end{tabular}

\end{table}



\begin{table}[!h]

\begin{tabular}{|p{60mm}p{100mm}|}

\hline

\multicolumn{2}{|l|}{\textbf{REQ\_0108:} Fréquence de service} \\ \hline

\emph{Type:} Métier & \emph{Liens:}  \\

\emph{Origine:}  & \emph{Validé:} Non \\

\emph{Version:} Initial & \emph{Test:}  \\

\emph{Priorité:} Must & \\ \hline

\multicolumn{2}{|p{16cm}|}{Le système doit proposer un plat selon la fréquence de service de ce plat (exemple 4 fois tous les 20 repas).} \\ \hline

\end{tabular}

\end{table}



\begin{table}[!h]

\begin{tabular}{|p{60mm}p{100mm}|}

\hline

\multicolumn{2}{|l|}{\textbf{REQ\_0410:} Composants des repas} \\ \hline

\emph{Type:} Métier & \emph{Liens:}  \\

\emph{Origine:}  & \emph{Validé:} Non \\

\emph{Version:} Initial & \emph{Test:}  \\

\emph{Priorité:} Must & \\ \hline

\multicolumn{2}{|p{16cm}|}{Le système doit permettre d'ajouter et de supprimmer des éléments dans les composants des repas.} \\ \hline

\end{tabular}

\end{table}



\begin{table}[!h]

\begin{tabular}{|p{60mm}p{100mm}|}

\hline

\multicolumn{2}{|l|}{\textbf{REQ\_0411:} Listes par défaut} \\ \hline

\emph{Type:} Métier & \emph{Liens:}  \\

\emph{Origine:}  & \emph{Validé:} Non \\

\emph{Version:} Initial & \emph{Test:}  \\

\emph{Priorité:} Should & \\ \hline

\multicolumn{2}{|p{16cm}|}{Le système doit permettre de revenir aux listes par défaut recommandé par le gouvernement.} \\ \hline

\end{tabular}

\end{table}



\begin{table}[!h]

\begin{tabular}{|p{60mm}p{100mm}|}

\hline

\multicolumn{2}{|l|}{\textbf{REQ\_0500:} Fiche de commande} \\ \hline

\emph{Type:} Métier & \emph{Liens:}  \\

\emph{Origine:}  & \emph{Validé:} Non \\

\emph{Version:} Initial & \emph{Test:}  \\

\emph{Priorité:} Could & \\ \hline

\multicolumn{2}{|p{16cm}|}{Le système doit permettre, une fois les menus élaborés de générer un fiche de commande au format : à définir.} \\ \hline

\end{tabular}

\end{table}



\begin{table}[!h]

\begin{tabular}{|p{60mm}p{100mm}|}

\hline

\multicolumn{2}{|l|}{\textbf{REQ\_0501:} Publication menus} \\ \hline

\emph{Type:} Non Fonctionnelle & \emph{Liens:}  \\

\emph{Origine:}  & \emph{Validé:} Non \\

\emph{Version:} Initial & \emph{Test:}  \\

\emph{Priorité:} Could & \\ \hline

\multicolumn{2}{|p{16cm}|}{Le système doit permettre d'afficher les menus sur un site internet.} \\ \hline

\end{tabular}

\end{table}



\begin{table}[!h]

\begin{tabular}{|p{60mm}p{100mm}|}

\hline

\multicolumn{2}{|l|}{\textbf{REQ\_0600:} Validation des repas} \\ \hline

\emph{Type:} Contrainte & \emph{Liens:}  \\

\emph{Origine:}  & \emph{Validé:} Non \\

\emph{Version:} Initial & \emph{Test:}  \\

\emph{Priorité:} Must & \\ \hline

\multicolumn{2}{|p{16cm}|}{Le système doit gérer un cycle de validation des repas (en cours d'élaboration, en attente de validation, validé.} \\ \hline

\end{tabular}

\end{table}



\begin{table}[!h]

\begin{tabular}{|p{60mm}p{100mm}|}

\hline

\multicolumn{2}{|l|}{\textbf{REQ\_0601:} Droits utilisateurs} \\ \hline

\emph{Type:} Contrainte & \emph{Liens:}  \\

\emph{Origine:}  & \emph{Validé:} Non \\

\emph{Version:} Initial & \emph{Test:}  \\

\emph{Priorité:} Must & \\ \hline

\multicolumn{2}{|p{16cm}|}{Le système doit permettre de gérer différent droit selon le type d'utilisateur.} \\ \hline

\end{tabular}

\end{table}



\begin{table}[!h]

\begin{tabular}{|p{60mm}p{100mm}|}

\hline

\multicolumn{2}{|l|}{\textbf{REQ\_0700:} Menus à assembler} \\ \hline

\emph{Type:} Métier & \emph{Liens:}  \\

\emph{Origine:}  & \emph{Validé:} Non \\

\emph{Version:} Initial & \emph{Test:}  \\

\emph{Priorité:} Must & \\ \hline

\multicolumn{2}{|p{16cm}|}{Le système doit afficher les menu à assembler pour un jour donnée et émettre une étiquette au format : à définir.} \\ \hline

\end{tabular}

\end{table}



\begin{table}[!h]

\begin{tabular}{|p{60mm}p{100mm}|}

\hline

\multicolumn{2}{|l|}{\textbf{REQ\_0701:} Limite prix repas} \\ \hline

\emph{Type:} Contrainte & \emph{Liens:}  \\

\emph{Origine:}  & \emph{Validé:} Non \\

\emph{Version:} Initial & \emph{Test:}  \\

\emph{Priorité:} Must & \\ \hline

\multicolumn{2}{|p{16cm}|}{Le système doit permettre de fixer une limite au prix d'un repas.} \\ \hline

\end{tabular}

\end{table}



\begin{table}[!h]

\begin{tabular}{|p{60mm}p{100mm}|}

\hline

\multicolumn{2}{|l|}{\textbf{REQ\_0702:} Prix repas} \\ \hline

\emph{Type:} Métier & \emph{Liens:}  \\

\emph{Origine:}  & \emph{Validé:} Non \\

\emph{Version:} Initial & \emph{Test:}  \\

\emph{Priorité:} Must & \\ \hline

\multicolumn{2}{|p{16cm}|}{Le système doit permettre de renseigner le prix des éléments d'un repas.} \\ \hline

\end{tabular}

\end{table}



\begin{table}[!h]

\begin{tabular}{|p{60mm}p{100mm}|}

\hline

\multicolumn{2}{|l|}{\textbf{REQ\_0902:} Profil patient} \\ \hline

\emph{Type:} Métier & \emph{Liens:}  \\

\emph{Origine:}  & \emph{Validé:} Non \\

\emph{Version:} Initial & \emph{Test:}  \\

\emph{Priorité:} Must & \\ \hline

\multicolumn{2}{|p{16cm}|}{Le système doit permettre de renseigner un profil patient comportant les éléments suivants :  
- regime particulier : liste à définir  
- allergie : liste à définir  
- contre-indication : liste à définir.} \\ \hline

\end{tabular}

\end{table}



\begin{table}[!h]

\begin{tabular}{|p{60mm}p{100mm}|}

\hline

\multicolumn{2}{|l|}{\textbf{REQ\_1000:} État civil} \\ \hline

\emph{Type:} Métier & \emph{Liens:}  \\

\emph{Origine:}  & \emph{Validé:} Non \\

\emph{Version:} Initial & \emph{Test:}  \\

\emph{Priorité:} Must & \\ \hline

\multicolumn{2}{|p{16cm}|}{Le système doit permettre de renseigner l'état civil d'un patient.} \\ \hline

\end{tabular}

\end{table}



\begin{table}[!h]

\begin{tabular}{|p{60mm}p{100mm}|}

\hline

\multicolumn{2}{|l|}{\textbf{REQ\_1001:} Localisation patient} \\ \hline

\emph{Type:} Métier & \emph{Liens:}  \\

\emph{Origine:}  & \emph{Validé:} Non \\

\emph{Version:} Initial & \emph{Test:}  \\

\emph{Priorité:} Must & \\ \hline

\multicolumn{2}{|p{16cm}|}{Le système doit permettre de renseigner la localisation particulière d'un patient.} \\ \hline

\end{tabular}

\end{table}



\begin{table}[!h]

\begin{tabular}{|p{60mm}p{100mm}|}

\hline

\multicolumn{2}{|l|}{\textbf{REQ\_1002:} Grammages} \\ \hline

\emph{Type:} Métier & \emph{Liens:}  \\

\emph{Origine:}  & \emph{Validé:} Non \\

\emph{Version:} Initial & \emph{Test:}  \\

\emph{Priorité:} Must & \\ \hline

\multicolumn{2}{|p{16cm}|}{Le système doit permettre de gérer les grammage de plat.} \\ \hline

\end{tabular}

\end{table}



\begin{table}[!h]

\begin{tabular}{|p{60mm}p{100mm}|}

\hline

\multicolumn{2}{|l|}{\textbf{REQ\_1003:} Plateaux repas} \\ \hline

\emph{Type:} Métier & \emph{Liens:}  \\

\emph{Origine:}  & \emph{Validé:} Non \\

\emph{Version:} Initial & \emph{Test:}  \\

\emph{Priorité:} Must & \\ \hline

\multicolumn{2}{|p{16cm}|}{Le système doit pouvoir gérer des plateaux repas de type : sans régime particulier ou avec régime particulier.} \\ \hline

\end{tabular}

\end{table}



\begin{table}[!h]

\begin{tabular}{|p{60mm}p{100mm}|}

\hline

\multicolumn{2}{|l|}{\textbf{REQ\_1004:} Groupes} \\ \hline

\emph{Type:} Métier & \emph{Liens:}  \\

\emph{Origine:}  & \emph{Validé:} Non \\

\emph{Version:} Initial & \emph{Test:}  \\

\emph{Priorité:} Should & \\ \hline

\multicolumn{2}{|p{16cm}|}{Le système doit gérer les patients par groupes selon leur régime, exemple le groupe des intolérant au lactose.} \\ \hline

\end{tabular}

\end{table}



\begin{table}[!h]

\begin{tabular}{|p{60mm}p{100mm}|}

\hline

\multicolumn{2}{|l|}{\textbf{REQ\_1005:} Génération automatique} \\ \hline

\emph{Type:} Fonctionnelle & \emph{Liens:}  \\

\emph{Origine:}  & \emph{Validé:} Non \\

\emph{Version:} Initial & \emph{Test:}  \\

\emph{Priorité:} Must & \\ \hline

\multicolumn{2}{|p{16cm}|}{Le système doit permettre de générer automatiquement les repas pour un groupe de patients particulier.} \\ \hline

\end{tabular}

\end{table}




\section{Analyse fonctionnelle}

\subsection{Cas d'utilisation}

Les acteurs humains pour le système Vitameal sont les suivants :

Le diététicien : la personne en charge de l'élaboration des menus servis aux patients. Pour cela il doit pouvoir composer les menus des 3 repas journaliers selon les contraintes médicales de chaque patients. Il peut remplir lui-même les profils diététiques des patients mais ceux-ci doivent être validé par un médecin.

Le médecin : la personne en charge du dossier médical des patients, qui valide les profils diététiques remplit par les diététiciens.
 
Le service restauration : Les personnes en charge de la préparation et de la commande des repas.

\begin{figure}[H]
\centering
\includegraphics[scale=0.8]{../../CasDUtilisations/diagramme_cas_utilisation.png}
\caption{Diagramme des cas d'utilisation principal}
\end{figure}

Le stéréotype ``secondaire'' dans le diagramme des cas d'utilisation principal indique que le cas d'utilisation ne fait pas partie des cas d'utilisation principaux et qu'il n'est pas obligatoire pour que le système fonctionne.

\subsection{Description des cas d'utilisation}

\subsubsection{Préparer les menus}

\noindent\textbf{Nom :} Préparer les menus \\
\textbf{ID :} UC101 \\
\textbf{Description :} Le service restauration souhaite pouvoir préparé les menus par groupe de patient. \\
\textbf{Auteur :} Nicolas SYMPHORIEN \\
\textbf{Dates(s) :} 12/06/2017 \\
\textbf{Acteurs :} Le service restauration \\
\textbf{Pré-condition :} L'utilisateur a consulté les menus d'un groupe de patient (voir cas d'utilisation ''consulter les menus`` ).

\noindent \textbf{Scénario principal : } Figure \ref{PreparerMenuSeq}

\begin{enumerate}
	\item Le service restauration choisi les plat d'un jour qu'il veut préparer.
	\item Le système affiche la composition des plat du jour choisi avec les quantités pour chaque ingrédients
	\item Le service restauration peut choisir d'élaborer les plats d'un autre jour , dans ce cas le cas d'utilisation reprend à l'étape 1, sinon le cas d'utilisation se termine.
\end{enumerate}

\noindent \textbf{Post-Conditions:} Le service restauration a préparé tous les plats des jours qu'il souhaite.

\begin{figure}
\centering
\includegraphics{../../CasDUtilisations/PreparerMenus/sequence_preparer_menus.png}
\caption{Diagramme de séquence du cas d'utilisation préparer les menus}
\label{PreparerMenuSeq}
\end{figure}

\subsubsection{Consulter les menus}

\noindent\textbf{Nom :} Consulter les menus \\
\textbf{ID :} UC102 \\
\textbf{Description :} Le service restauration souhaite pouvoir consulté les menus par groupe de patient. \\
\textbf{Auteur :} Nicolas SYMPHORIEN \\
\textbf{Dates(s) :} 11/06/2017 \\
\textbf{Acteurs :} Le service restauration \\
\textbf{Pré-condition :} L'utilisateur doit être identifié.

\noindent \textbf{Scénario principal :} Figure \ref{ConsulterMenusSeq}

\begin{enumerate}
	\item \label{UC102_step1}Le service restauration choisi le groupe de patient pour lequel il veut voir le menu.
	\item \label{UC102_step2}Le système affiche le menu de la semaine en cours selon le groupe de patient choisi
	\item \label{UC102_step3}Le service restauration peut choisir de consulter le menu pour un autre groupe de patient, dans ce cas le cas d'utilisation reprend à l'étape \ref{UC102_step1}, sinon le cas d'utilisation se termine.
\end{enumerate}

\noindent \textbf{Scénario alternatif :}

\textit{Premier scénario alternatif :}
Le scénario alternatif suivant débute après l'étape \ref{UC102_step2} du scénario nominal
\begin{enumerate}
	\item L'utilisateur peut changer de groupe de patient et le cas d'utilisation reprend à l'étape \ref{UC102_step2} du scénario nominal
\end{enumerate}

\textit{Second scénario alternatif :}
Le scénario alternatif suivant débute après l'étape \ref{UC102_step2} du scénario nominal
\begin{enumerate}
	\item Le service restauration peut changer de semaine
	\item Le système affiche le menu de la semaine choisi si il existe, sinon il affiche un message. 
\end{enumerate}
Le cas d'utilisation reprend à l'étape \ref{UC102_step3} du scénario nominal.

\noindent \textbf{Post-Conditions:} Le menu est affiché pour le groupe de patient choisi.

\begin{figure}
\centering
\includegraphics[scale=0.75]{../../CasDUtilisations/ConsulterMenus/sequence_consulter_menus.png}
\caption{Diagramme de séquence du cas d'utilisation consulter les menus}
\label{ConsulterMenusSeq}
\end{figure}

\subsubsection{Composer un plat}

\noindent \textbf{Nom:} Composer un plat \\
\textbf{ID:} UC401\\
\textbf{Description :} Le diététicien souhaite pouvoir composé un plat (petit-déjeuner, déjeuner, souper) en renseignant sa composition.\\
\textbf{Auteurs :} Nicolas SYMPHORIEN\\
\textbf{Date :} 16/06/2017 \\
\textbf{Acteurs :} Le diététicien \\
\textbf{Pré-condition :} \\
Le diététicien doit être connecter (Voir le cas d'utilisation secondaire ``S'authentifier''). \\
La liste des plats doit être accessible.

\noindent \textbf{Scénario principal : } Figure \ref{ComposerPlatSeq}

\begin{enumerate}
	\item \label{UC401_step1}Le système affiche la liste des plats déjà crée.
	\item \label{UC401_step2}Le diététicien choisi de créer un nouveau plat.
	\item Le système affiche une page permettant d'entrer les ingrédients composant le plat ainsi que leurs quantités
	\item Le diététicien choisi les ingrédients qu'il veut mettre dans son plat
	\item Le système enregistre le plat crée et affiche un message de confirmation de création
\end{enumerate}

 \noindent \textbf{Scénario alternatif :}

Les deux scénario alternatifs débute après l'étape \ref{UC401_step1} du scénario nominal.
\begin{enumerate}
	\item Le diététicien choisi de modifier un plat déjà existant.
	\begin{enumerate}
		\item Le système affiche les ingrédients du plat à modifier
		\item Le diététicien modifie la composition du plat et confirme les modifications
		\item Le système enregistre le plat modifié et affiche un message de confirmation de modification
	\end{enumerate}
	\item Le diététicien choisi de supprimer un plat déjà existant.
	\begin{enumerate}
		\item Le système affiche un message d'avertissement avant la suppression
		\item L'utilisateur confirme la suppression du plat
		\item Le système supprime le plat modifié et affiche un message de confirmation de suppression
	\end{enumerate}
\end{enumerate}
Dans les deux cas, le cas d'utilisation reprend à l'étape \ref{UC401_step2} du scénario nominal.

\noindent \textbf{Post-Conditions:} Le plat est crée, modifié ou supprimé.

\begin{figure}
\centering
\includegraphics[scale=0.3]{../../CasDUtilisations/CompositionPlat/sequence_UC_ComposerPlat.png}
\caption{Diagramme de séquence du cas d'utilisation composer un plat}
\label{ComposerPlatSeq}
\end{figure}

%-*- coding: utf-8 -*-
\subsection{Génération automatique des menus}
\begin{figure}[H]
\label{MenuGen}
  \centering
      \includegraphics[width=1.00\textwidth]{../../CasDUtilisations/MenuGen/UseCase_Diagram} %
\caption{Génération automatique des menus}
\end{figure}

\begin{description}
\item[Nom:] Génération automatique des menus.
\item[ID:] REQ\_01000, REQ\_10020, REQ\_10050, REQ\_01010, REQ\_01020, REQ\_01030, REQ\_01040, REQ\_01050.
\item[Description:] Permet la génération automatique des menus.
\item[Acteurs:] Diététiciens.
\item[Pré-Conditions:] Le diététicien s'est connecté au système.
\item[Scénario principal:] Le diététicien sélectionne le groupe de patients pour lequel il veut générer les menus, ensuite il lance la génération automatique.
\item[Post-Conditions:] Les menus sont générés.
\item[Scénario alternatif:] Aucun.
\end{description}


%-*- coding: utf-8 -*-
\subsubsection{Renseigner les profils patients}

\begin{figure}[!h]
  \label{diagramme-renseigner-les-profils-patients}
  \centering
  \includegraphics[width=0.9\textwidth]{../../CasDUtilisations/ProfilPatient/UseCaseProfilPatient.png}
  \caption{Use case renseigner les profils patients}
\end{figure}

\textbf{UC400 - Renseigner profil patient}
\begin{description}
\item [Nom :] Renseigner profil patient
\item [ID :] UC400
\item [Description :] Le diététicien souhaite pouvoir renseigner un profil patient.
\item [Auteur :] Sonia OTHMANI
\item [Date :] 08/05/2017
\item [Acteurs :] Le diététicien
\item [Pré-condition :] L’utilisateur doit être identifié en tant que diététicien (Voir cas d’utilisation \enquote{S’authentifier})
\item [Scénario principal :]
  \begin{enumerate}
  \item Le système affiche une page permettant de créer un profil patient.
  \item L’utilisateur complète les champs relatifs au patient : nom, prénom, date de naissance, sexe, allergies, contre-indications, affections, service d'hospitalisation et numéro de chambre.
  \item L’utilisateur enregistre le profil patient.
 
  \end{enumerate}
\item [Scénario alternatif :] Aucun
\item [Post-Conditions :] Le profil patient est créé et enregistré.
\end{description}

\textbf{UC401 - Valider le profil patient}
\begin{description}
\item [Nom :] Valider le profil patient
\item [ID :] UC401
\item [Description :] Le médecin souhaite pouvoir valider un profil patient.
\item [Auteur :] Sonia OTHMANI
\item [Date :] 08/05/2017
\item [Acteurs :] Le diététicien
\item [Pré-condition :] L’utilisateur doit être identifié en tant que médecin (Voir cas d’utilisation \enquote{S’authentifier})
\item [Scénario principal :]
  \begin{enumerate}
  \item Le système affiche une liste de profils patients en attente de validation.
  \item L’utilisateur sélectionne la fiche patient à valider.
  \item L’utilisateur consulte le profil patient.
  \item L’utilisateur valide le profil patient.
  \item Le système enregistre le profil patient comme validé.
  \end{enumerate}
\item [Scénario alternatif :] Aucun
\item [Post-Conditions :] Le profil patient est validé.
\end{description}


%-*- coding: utf-8 -*-
\section{Backlog}
\subsection{Profil patient}
\begin{enumerate}
\item En tant que diététicien, je souhaite pouvoir créer un profil patient.
\item En tant que diététicien, je souhaite pouvoir renseigner un profil patient comportant les éléments suivants : nom, prénom, date de naissance.
\item En tant que diététicien, je souhaite pouvoir renseigner des allergies éventuelles dans le profil patient.
\item En tant que diététicien, je souhaite pouvoir renseigner un régime alimentaire prescrit, en choisissant entre les items suivants : régime sans sel, régime sans sucre, régime sans matières grasses, régime sans gluten, régime sans lactose.
\item En tant que diététicien, je souhaite pouvoir modifier un profil patient.
\item En tant que diététicien, je souhaite pouvoir supprimer un profil patient.
\item En tant que diététicien, je souhaite pouvoir gérer les patients par groupes selon leur régime prescrit, exemple le groupe des intolérants au lactose.
\item En tant que médecin, je souhaite pouvoir valider le profil diététique du patient, en plus d'effectuer toutes les actions réalisables par le diététicien.
\end{enumerate}

\subsection{Composition des plats}
\begin{enumerate}
\item En tant que diététicien, je souhaite pouvoir ajouter des plats et leurs définitions dans la listes des plats pouvant être préparés.
\item En tant que diététicien, je souhaite pouvoir descrire un plat avec sa liste d'ingrédients et les quantités nécessaires à sa réalisation.
\item Le système doit proposer un plat selon la fréquence de service de ce plat (exemple 4 fois tous les 20 repas).
  \end{enumerate}

