%-*- coding: utf-8 -*-
\section{Backlog}
\subsection{Profil patient}
\begin{enumerate}
\item En tant que diététicien, je souhaite pouvoir créer un profil patient.
\item En tant que diététicien, je souhaite pouvoir renseigner un profil patient comportant les éléments suivants : nom, prénom, date de naissance.
\item En tant que diététicien, je souhaite pouvoir renseigner des allergies éventuelles dans le profil patient.
\item En tant que diététicien, je souhaite pouvoir renseigner un régime alimentaire prescrit, en choisissant entre les items suivants : régime sans sel, régime sans sucre, régime sans matières grasses, régime sans gluten, régime sans lactose.
\item En tant que diététicien, je souhaite pouvoir modifier un profil patient.
\item En tant que diététicien, je souhaite pouvoir supprimer un profil patient.
\item En tant que diététicien, je souhaite pouvoir gérer les patients par groupes selon leur régime prescrit, exemple le groupe des intolérants au lactose.
\item En tant que médecin, je souhaite pouvoir valider le profil diététique du patient, en plus d'effectuer toutes les actions réalisables par le diététicien.
\end{enumerate}

\subsection{Composition des plats}
\begin{enumerate}
\item En tant que diététicien, je souhaite pouvoir ajouter des plats et leurs définitions dans la listes des plats pouvant être préparés.
\item En tant que diététicien, je souhaite pouvoir descrire un plat avec sa liste d'ingrédients et les quantités nécessaires à sa réalisation.
\item Le système doit proposer un plat selon la fréquence de service de ce plat (exemple 4 fois tous les 20 repas).
  \end{enumerate}
